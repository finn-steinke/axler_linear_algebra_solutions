\documentclass{article}
\usepackage{graphicx}
\usepackage{amsmath,amsthm,amssymb}
\usepackage{mathtools}
\usepackage{xcolor}
\usepackage{mdframed}
\usepackage[margin=1in]{geometry}


\begin{document}

\begin{titlepage}
   \begin{center}
       \vspace*{1cm}

       \textbf{Worked Solutions for \textit{Linear Algebra Done Right}}

       \vspace{0.5cm}
        Dr. Sheldon Axler, Fourth Edition
            
       \vspace{0.8cm}

       \text{Answers provided by Finn Steinke}

       \vfill
            
       
            
       \vspace{0.8cm}
     
    
            
       
            
   \end{center}
\end{titlepage}


\section*{Chapter 1}

\subsection*{Exercises 1A}

\textbf{1} \hspace{3 mm} Show that $\alpha + \beta = \beta + \alpha$ for all $\alpha, \beta \in \mathbb{C}$. \color{red}

\begin{proof}
    $\forall \alpha, \beta \in \mathbb{C}$, suppose $\alpha \stackrel{\text{def}}{=} a + bi, \  \beta \stackrel{\text{def}}{=} c + di$, where $a, b, c, d \in \mathbb{R}$.\\
    \indent Then
    \begin{align*}
        \alpha + \beta &= (a + bi) + (c + di) &&\text{(by definition)}\\
            &= (a + c) + (b + d)i &&\text{(by definition of addition over $\mathbb{C}$)}\\
            &= (c + a) + (d + b)i &&\text{(holds for commutativity over $\mathbb{R}$)}\\
            &= (c + di) + (a + bi) &&\text{(by definition of addition over $\mathbb{C}$)}\\
            &= \beta + \alpha &&\text{(by definition)}
    \end{align*}
    $\implies$Thus $\alpha + \beta = \beta + \alpha \hspace{2 mm} \forall \alpha, \beta \in \mathbb{C}$. 
    
\end{proof}
\color{black}

\noindent \textbf{2} \hspace{3 mm} Show that $(\alpha + \beta) + \lambda = \alpha + (\beta + \lambda)$ for all $\alpha, \beta, \lambda \in \mathbb{C}$. \color{red}

\begin{proof}
    $\forall \alpha, \beta, \lambda \in \mathbb{C}$, suppose $\alpha \stackrel{\text{def}}{=} a + bi, \  \beta \stackrel{\text{def}}{=} c + di, \ \lambda \stackrel{\text{def}}{=} j + ki$, where $a, b, c, d, j, k \in \mathbb{R}$.\\
    \indent Then
    \begin{align*}
        (\alpha + \beta) + \lambda &= \Big((a + bi) + (c + di)\Big) + (j + ki) &&\text{(by definition)}\\
        &= \Big((a + c) + (b + d)i\Big) + (j + ki) &&\text{(by definition of addition over $\mathbb{C}$)}\\
        &= \Big((a + c) + j\Big) + \Big((b + d) + k\Big)i &&\text{(by definition of addition over $\mathbb{C}$)}\\
        &= \Big(a + (c + j)\Big) + \Big(b + (d + k)\Big)i &&\text{(holds for associativity over $\mathbb{R}$)}\\
        &= (a + bi) + \Big((c + j) + (d + k)i\Big) &&\text{(by definition of addition over $\mathbb{C}$)}\\
        &= (a + bi) + \Big((c + di) + (j + ki)\Big) &&\text{(by definition of addition over $\mathbb{C}$)}\\
        &= \alpha + (\beta + \lambda) &&\text{(by definition)}
    \end{align*}
    $\implies$Thus $(\alpha + \beta) + \lambda = \alpha + (\beta + \lambda) \hspace{2 mm} \forall \alpha, \beta, \lambda \in \mathbb{C}$.
\end{proof}
\color{black}
\pagebreak

\noindent \textbf{3} \hspace{3 mm} Show that $(\alpha\beta)\lambda = \alpha(\beta\lambda)$ for all $\alpha, \beta, \lambda \in \mathbb{C}$. \color{red}

\begin{proof}
    $\forall \alpha, \beta, \lambda \in \mathbb{C}$, suppose $\alpha \stackrel{\text{def}}{=} a + bi, \  \beta \stackrel{\text{def}}{=} c + di, \ \lambda \stackrel{\text{def}}{=} j + ki$, where $a, b, c, d, j, k \in \mathbb{R}$.\\
    \indent Then
    \begin{align*}
        (\alpha\beta)\lambda &= \Big((a + bi)(c + di)\Big)(j + ki) &&\text{(by definition)}\\
        &= \Big((ac - bd) + (ad + bc)i\Big)(j + ki) &&\text{(by definition of multiplication over $\mathbb{C}$)}\\
        &= \Big((ac - bd)j - (ad + bc)k\Big) + \Big((ac - bd)k + (ad + bc)j\Big)i &&\text{(by definition of multiplication over $\mathbb{C}$)}\\
        &= (acj - bdj - adk - bck) + (ack - bdk + adj + bcj)i &&\text{(holds for distributivity over $\mathbb{R}$)}\\
        &= (acj - adk - bdj - bck) + (ack + adj - bdk + bcj)i &&\text{(holds for commutativity over $\mathbb{R}$)}\\
        &= \Big(a(cj - dk) - b(dj + ck)\Big) + \Big(a(ck + dj) + b(-dk + cj)\Big)i &&\text{(holds for distributivity over $\mathbb{R}$)}\\
        &= (a + bi)\Big((cj - dk) + (ck + dj)i\Big) &&\text{(by definition of multiplication over $\mathbb{C}$)}\\
        &= (a + bi)\Big((c + di)(j + ki)\Big) &&\text{(by definition of multiplication over $\mathbb{C}$)}\\
        &= \alpha(\beta\lambda) &&\text{(by definition)}\\
    \end{align*}
    $\implies$Thus $(\alpha\beta)\lambda = \alpha(\beta\lambda) \hspace{2 mm} \forall \alpha, \beta, \lambda \in \mathbb{C}$.

\end{proof}
\color{black}

\noindent \textbf{4} \hspace{3 mm} Show that $\lambda(\alpha + \beta) = \lambda\alpha + \lambda\beta$ for all $\lambda, \alpha, \beta \in \mathbb{C}$. \color{red}

\begin{proof}
    $\forall \lambda, \alpha, \beta \in \mathbb{C}$, suppose $\lambda \stackrel{\text{def}}{=} j + ki, \ \alpha \stackrel{\text{def}}{=} a + bi, \ \beta \stackrel{\text{def}}{=} c + di$, where $j, k, a, b, c, d \in \mathbb{R}$.\\
    \indent Then
    \begin{align*}
        \lambda(\alpha + \beta) &= (j + ki)\Big((a + bi) + (c + di)\Big) &&\text{(by definition)}\\
        &= (j + ki)\Big((a+c) + (b + d)i\Big) &&\text{(by definition of addition over $\mathbb{C}$)}\\
        &= \Big(j(a + c) - k(b + d)\Big) + \Big(j(b + d) + k(a + c)\Big)i &&\text{(by definition of multiplication over $\mathbb{C}$)}\\
        &= (ja + jc - kb - kd) + (jb + jd + ka + kc)i &&\text{(holds for distributivity over $\mathbb{R}$)}\\
        &= (ja - kb + jc - kd) + (jb + ka + jd + kc)i &&\text{(holds for commutativity over $\mathbb{R}$)}\\
        &= (ja - kb) + (jc - kd) + (jb + ka)i + (jd + kc)i &&\text{(holds for distributivity over $\mathbb{R}$)}\\
        &= \Big((ja - kb) + (jb + ka)i\Big) + \Big((jc - kd) + (jd + kc)i\Big) &&\text{(holds for commutativity over $\mathbb{R}$)}\\
        &= (j + ki)(a + bi) + (j + ki)(c + di) &&\text{(by definition of multiplication over $\mathbb{C}$)}\\
        &= \lambda\alpha + \lambda\beta &&\text{(by definition)}\\
    \end{align*}
    $\implies$Thus $\lambda(\alpha + \beta) = \lambda\alpha + \lambda\beta \hspace{2 mm} \forall \lambda, \alpha, \beta \in \mathbb{C}$.
\end{proof}
\color{black}

\pagebreak

\noindent \textbf{5} \hspace{3 mm} Show that for every $\alpha \in \mathbb{C}$, there exists a unique $\beta \in \mathbb{C}$ such that $\alpha + \beta = 0$. \color{red}

\begin{proof}
    $\forall \alpha \in \mathbb{C}$, we assume $\exists! \beta \in \mathbb{C}$ such that $\alpha + \beta = 0$. To this end, suppose $\alpha \stackrel{\text{def}}{=} a + bi, \beta \stackrel{\text{def}}{=} c + di$, \\
    \indent where $a,b,c,d \in \mathbb{R}$. It follows that
    \begin{align*}
        &\hspace{9.75 mm} \alpha + \beta = 0 &&\text{(by assumption)}\\
        &\implies (a + bi) + (c + di) = 0 &&\text{(by definition)}\\
        &\implies (a + c) + (b + d)i = 0 &&\text{(by definition of addition over $\mathbb{C}$)}\\
        &\implies (a + (-a)) + (b + (-b))i = 0 &&\text{(holds for definition of additive inverse over $\mathbb{R}$)}\\
        &\implies (a + bi) + \big((-a) + (-b)i\big) = 0 &&\text{(by definition of addition over $\mathbb{C}$)}\\
        &\implies (a + bi) + (-1)(a + bi) = 0 &&\text{(holds for distributivity over $\mathbb{R}$)}\\
        &\implies \alpha + (-1)(\alpha) = 0 &&\text{(by definition)}\\
        &\implies \alpha + -\alpha = 0 &&\text{(by definition of multiplication over $\mathbb{C}$)}\\
    \end{align*}
    $\implies \forall \alpha \in \mathbb{C}, \ \exists! \beta = -\alpha \in \mathbb{C}$ such that $\alpha + (-\alpha) = 0$.

\end{proof}
\color{black}

\noindent \textbf{6} \hspace{3 mm} Show that for every $\alpha \in \mathbb{C}$ with $\alpha \neq 0$, there exists a unique $\beta \in \mathbb{C}$ such that $\alpha\beta = 1$. \color{red}

\begin{proof}
    $\forall \alpha \in \mathbb{C}$, we assume $\exists! \beta \in \mathbb{C}$ such that $\alpha\beta = 1$. To this end, suppose $\alpha \stackrel{\text{def}}{=} a + bi, \beta \stackrel{\text{def}}{=} c + di$, \\
    \indent where $a, b, c, d \in \mathbb{R}$ and $\alpha \neq 0$. If follows that
    \begin{align*}
        &\hspace{9.75 mm} \alpha\beta = 1 &&\text{(by assumption)}\\
        &\implies (a + bi)(c + di) =1 &&\text{(by definition)}\\
        &\implies (ac - bd) + (ad + bc)i = 1 &&\text{(by definition of multiplication over $\mathbb{C}$)}\\
    \end{align*}

    \begin{mdframed}[
    linecolor=red,        
    fontcolor=red         
]
        \indent For the above equality to hold, $\Re(\alpha\beta) = 1$, and $\Im(\alpha\beta) = 0$, giving us two equations\\
        \[ac - bd = 1 \tag{$\ast$}\]
        \[ad + bc = 0 \tag{$\ast\ast$}\]
        \indent Rearranging ($\ast$) for $c$ gives us
        \[c = \frac{1 + bd}{a}\]
        \indent Rearranging ($\ast\ast$) for $d$ gives us
        \[d = -\frac{bc}{a}\]
        \indent Substituting ($\ast$) into ($\ast\ast$) gives us
        \begin{align*}
            &\hspace{9.75 mm} d = -\frac{b\big(\frac{1 + bd}{a}\big)}{a} \\
            &\implies d = -\frac{b(1 + bd)}{a^{2}} \\
            &\implies da^{2} = - b(1 + bd) \\
            &\implies da^{2} = - b - b^{2}d \\
            &\implies da^{2} + b^{2}d = - b \\
            &\implies d(a^{2} + b^{2}) = - b \\
            &\implies d = -\frac{b}{a^{2} + b^{2}} \tag{$\ast\ast\ast$}
        \end{align*}
        \indent Substituting ($\ast\ast\ast$) into ($\ast$) gives us
        \begin{align*}
            c &= \frac{1 + b\big(\frac{-b}{a^{2} + b^{2}}\big)}{a} \\
            &= \frac{1 - \big(\frac{b^{2}}{a^{2} + b^{2}}\big)}{a} \\
            &= \frac{\big(\frac{a^{2} + b^{2}}{a^{2} + b^{2}}\big) - \big(\frac{b^{2}}{a^{2} + b^{2}}\big)}{a} \\
            &= \frac{\big(\frac{a^{2}}{a^{2} + b^{2}}\big)}{a} = \frac{a^{2}}{a(a^{2} + b^{2})} \\
            &= \frac{a}{a^{2} + b^{2}}
        \end{align*}
    \end{mdframed}
    
    \indent Now, using our new definitions of $c$ and $d$
    \begin{align*}
        &\hspace{9.75 mm} 1 = (ac - bd) + (ad + bc)i \\
        &\implies 1 = \Big(a\Big(\frac{a}{a^{2} + b^{2}}\Big) - b\Big(\frac{-b}{a^{2} + b^{2}}\Big)\Big) + \Big(a\Big(\frac{-b}{a^{2} + b^{2}}\Big) + b\Big(\frac{a}{a^{2} + b^{2}}\Big)\Big)i &&\text{(by definition)}\\
        &\implies 1 = \Big(\Big(\frac{a^{2}}{a^{2} + b^{2}}\Big) + \Big(\frac{b^{2}}{a^{2} + b^{2}}\Big)\Big) + \Big(\Big(\frac{-ab}{a^{2} + b^{2}}\Big) + \Big(\frac{ab}{a^{2} + b^{2}}\Big)\Big)i &&\text{(holds for multiplication over $\mathbb{R}$)}\\
        &\implies 1 = \Big(\frac{a^{2} + b^{2}}{a^{2} + b^{2}}\Big) = 1 &&\text{(holds for addition over $\mathbb{R}$)}\\
    \end{align*}
    \indent We can redefine $\beta$ in terms of $\alpha$
    \begin{align*}
        \beta &= \frac{a - bi}{a^{2} + b^{2}}\\
        \beta &\stackrel{\text{def}}{=} \frac{1}{\alpha}
    \end{align*}
    $\implies \forall \alpha \in \mathbb{C}, \ \exists! \beta = \dfrac{1}{\alpha} \in \mathbb{C}$ such that $\alpha\Big(\dfrac{1}{\alpha}\Big) = 1$.
    
\end{proof}
\color{black}

\pagebreak

\noindent \textbf{7} \hspace{3 mm} Show that $\dfrac{-1 + \sqrt{3}i}{2}$ is a cube root of 1 (meaning that its cube equals 1). \color{red}

\begin{proof}
    \begin{align*}
        \Big(\dfrac{-1 + \sqrt{3}i}{2}\Big)^{3} &= \Big(\dfrac{-1 + \sqrt{3}i}{2}\Big)\Big(\dfrac{-1 + \sqrt{3}i}{2}\Big)\Big(\dfrac{-1 + \sqrt{3}i}{2}\Big) &&\text{(by definition)}\\
        &= \Big(\Big(\frac{1}{4} - \frac{3}{4}\Big) + \Big(-\frac{\sqrt{3}}{4} - \frac{\sqrt{3}}{4}\Big)i\Big)\Big(\dfrac{-1 + \sqrt{3}i}{2}\Big) &&\text{(by definition of multiplication over $\mathbb{C}$)}\\
        &= \Big(\frac{-1 - \sqrt{3}i}{2}\Big)\Big(\frac{-1 + \sqrt{3}i}{2}\Big) &&\text{(holds for addition over $\mathbb{R}$)}\\
        &= \Big(\frac{1}{4} + \frac{3}{4}\Big) + \Big(-\frac{\sqrt{3}}{4} + \frac{\sqrt{3}}{4}\Big)i &&\text{(by definition of multiplication over $\mathbb{C}$)}\\
        &= 1 &&\text{(holds for addition over $\mathbb{R}$)}\\
    \end{align*}
    $\implies \dfrac{-1 + \sqrt{3}i}{2} = \sqrt[3]{1}$ .
    
\end{proof}
\color{black}

\noindent \textbf{8} \hspace{3 mm} Find two distinct square roots of $i$. \color{red}\\

\indent Let $\alpha \in \mathbb{C}$, where $\alpha \stackrel{\text{def}}{=} a + bi$, with $a, b \in \mathbb{R}$.
\begin{align*}
    &\hspace{9.75 mm}i = \alpha^{2} \\
    &\implies i = (a + bi)^{2} \\
    &\implies i = a^{2} + 2abi - b^{2} \\
    &\implies i = \Im(a^{2} + 2abi - b^{2}) \\
    &\implies i = 2abi \\
    &\implies 1 = 2ab \\
    &\implies \frac{1}{2} = ab \\
\end{align*}

\begin{center}

    \fcolorbox{red}{white}{
        \parbox{0.55\textwidth}{
            $\hspace{9.75 mm}\Re(\alpha^{2}) = 0 = a^{2} - b^{2} \implies a^{2} = b^{2} \implies a = \pm b$
        }
    }
    
\end{center}

\begin{align*}
    &\implies \frac{1}{2} = (a)^{2}\\
    &\implies a = \pm {\frac{1}{\sqrt{2}}} = \pm \frac{\sqrt{2}}{2}\\
    &\implies \sqrt{i} = \pm \alpha = \pm\Big(\frac{\sqrt{2} + \sqrt{2}i}{2}\Big)\\
\end{align*}
$\implies \sqrt{i} = \dfrac{-\sqrt{2} - \sqrt{2}i}{2}, \ \dfrac{\sqrt{2} + \sqrt{2}i}{2}$ .\\

\color{black}

\pagebreak

\noindent \textbf{9} \hspace{3 mm} Find $x \in \mathbb{R}^{4}$ such that $(4, -3, 1, 7) + 2x = (5, 9, -6, 8)$. \color{red}\\

\indent Let $x \in \mathbb{R}^{4}$, where $x \stackrel{\text{def}}{=} (x_{1}, x_{2}, x_{3}, x_{4})$, with $x_{1}, x_{2}, x_{3}, x_{4} \in \mathbb{R}$.

\begin{align*}
    (4, -3, 1, 7) + 2x &= (5, 9, -6, 8)\\
    (4, -3, 1, 7) + 2(x_{1}, x_{2}, x_{3}, x_{4}) &= (5, 9, -6, 8)\\
    2(x_{1}, x_{2}, x_{3}, x_{4}) &= (5, 9, -6, 8) - (4, -3, 1, 7)\\
    2(x_{1}, x_{2}, x_{3}, x_{4}) &= (1, 12, -5, 1)\\
    (x_{1}, x_{2}, x_{3}, x_{4}) &= \frac{1}{2}(1, 12, -5, 1)\\
    (x_{1}, x_{2}, x_{3}, x_{4}) &= \Big(\frac{1}{2}, 6, -\frac{5}{2}, \frac{1}{2}\Big)\\
\end{align*}
$\implies x = \Big(\dfrac{1}{2}, 6, -\dfrac{5}{2}, \dfrac{1}{2}\Big)$ .\\\\

\color{black}

\noindent \textbf{10} \hspace{3 mm} Explain why there does not exist $\lambda \in \mathbb{C}$ such that $\lambda(2 - 3i, 5 + 4i, -6 + 7i) = (12 - 5i, 7 + 22i, -32 - 9i)$.\color{red}\\

\noindent We assume, for the sake of contradiction, that there exists $\lambda \in \mathbb{C}$ such that the equation above holds. This,
    \begin{align*}
        \lambda(2 - 3i) &= 12 - 5i &&\text{(by the definition of scalar multiplication)} \\
        \lambda &= \frac{12 - 5i}{2 - 3i} &&\text{(solving for $\lambda$)} \\
        &= \frac{(12 - 5i)(2 + 3i)}{(2 - 3i)(2 + 3i)} &&\text{(by the definition of division over $\mathbb{C}$)} \\
        &= \frac{24 + 36i - 10i - 15}{4 + 9} &&\text{(holds for distributivity over $\mathbb{C}$)} \\
        &= \frac{9 + 26i}{13} &&\text{(holds for commutativity over $\mathbb{C}$)} \\
        &= \frac{9}{13} + \frac{26}{13}i. &&\text{(separating $\Re({z})$ and $\Im({z})$ parts)}
    \end{align*}

    Substituting $\lambda = \frac{9}{13} + \frac{26}{13}i$ into the second equation yields
    \begin{align*}
        \lambda(5 + 4i) &= 7 + 22i &&\text{(by the definition of scalar multiplication)} \\
        \Big(\frac{9}{13} + \frac{26}{13}i\Big)(5 + 4i) &= 7 + 22i &&\text{(substitute $\lambda$)} \\
        &= \frac{1}{13}\Big(45 + 36i + 130i - 104\Big) &&\text{(by the definition of multiplication over $\mathbb{C}$)} \\
        &= \frac{1}{13}\Big(-59 + 166i\Big) &&\text{(by commutativity over $\mathbb{C}$)} \\
        &= -\frac{59}{13} + \frac{166}{13}i. &&\text{(separating $\Re({z})$ and $\Im({z})$ parts)}
    \end{align*}

    Comparing with $7 + 22i$, we observe a contradiction, as
    \[
    -\frac{59}{13} \neq 7 \quad \text{and} \quad \frac{166}{13} \neq 22.
    \]

    Therefore, no such $\lambda \in \mathbb{C}$ can exist.

\color{black}

\end{document}
